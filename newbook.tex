\documentclass{article}
\usepackage[utf8]{inputenc}
\usepackage{thumbpdf}
\usepackage[pdftex,
        colorlinks=true,
        urlcolor=rltblue,       % \href{...}{...} external (URL)
        filecolor=rltgreen,     % \href{...} local file
        linkcolor=rltred,       % \ref{...} and \pageref{...}
        pdftitle={Matemática Aplicada con Python},
        pdfauthor={José J. Reyna},
        pdfsubject={Just a test},
        pdfkeywords={programming, python, mathematics},
        pdfproducer={pdfLaTeX},
        pdfadjustspacing=1,
        pagebackref,
        pdfpagemode=None,
        bookmarksopen=true]{hyperref}
\usepackage{color}
\definecolor{rltred}{rgb}{0.75,0,0}
\definecolor{rltgreen}{rgb}{0,0.5,0}
\definecolor{rltblue}{rgb}{0,0,0.75}

\title{Matemática Aplicada con Python}
\author{José J. Reyna}
\date{\today}

\begin{document}\label{start}

\maketitle

\newpage

\section{Introducción}\label{introduccion}

\subsection{Misión}\label{mision}

La misión de éste libro es poner a disposición de cualquier persona dos de las principales
herramientas que hoy en día permiten conocer la realidad, realizar nuevos 
descubrimientos, construir empresas y tener poder militar. Esas son la matemática
y la programación. Justificar semejante misión (y convencerte de que es posible) 
es la tarea que ahora tiene el libro. Las siguientes partes, "breve historia de la
computación" y "propósito y estructura del libro", espero permitan ver la motivación y hacia donde
se quiere llegar.

\subsection{Breve historia de la computación}\label{historia}

Hoy en día las computadoras están en todas partes, cumpliendo diferentes
funciones y con distintas apariencias. En cada oficina podemos estar seguros de
que hay por lo menos una PC con algún procesador de textos y
un hoja de cálculo para crear documentos, y en cada escuela un grupo de niños al que
probablemente les están mintiendo al hacerles creer que usar esos programas es "informática". 
Pero los computadores no sólo están en los escritorios. Los smartphones e incluso 
las cónsolas de videojuegos tienen la misma estructura 
interna (circuitos, chips, procesadores...), y, como podría esperarse, tienen a
su vez el mismo origen.
\newline\newline
Por el año 1812, a Charles Babbage se le ocurrió la idea de sustituir a los
computistas humanos que creaban tablas de logaritmos (largas hojas llenas de números) 
por máquinas, probablemente porque le parecían demasiado lentos. Ésta iniciativa, sin embargo, tenía la pequeña limitación de que 
en aquella época no existían Microsoft, Apple o Linux, y ni siquiera los chips. 
Lo único que Babagge tenía a su disposición eran engranajes mecánicos. En base a eso
realizó varios diseños que nunca pudo ver terminados, pero que gracias a ellos hoy
se le considera el padre de la computación. Interesantemente, a finales del siglo XX 
la humanidad \href{http://www.economist.com/node/324654?story_id=E1_PNQGVQ}{se percató} de que en ciertas condiciones hostiles, como el espacio exterior y reactores nucleares, diseños similares
a los de Babbage presentan ventajas frente a la electrónica. Ciertamente debe estar descanzando en paz, a pesar de nunca haber visto uno de sus diseños funcionando. Lo principal de ésta parte del cuento para éste libro, en resumen, es que podemos afirmar que \emph{el primer ser humano que se imaginó una
computadora quería hacer cálculos matemáticos más rápido}.
\newline\newline
Si nos saltamos unos cuantos años y llegamos a la segunda guerra mundial, nos 
encontramos con que los alemanes usaban 
\href{https://en.wikipedia.org/wiki/Enigma_machine}{una nieta de Babbage} 
para encriptar sus mensajes secretos. Y, como era de esperarse, los aliados no 
estaban del todo contentos con esa situación. Hombres como 
\href{https://es.wikipedia.org/wikiJohn_von_Neumann}{John von Neumann} 
y 
\href{https://es.wikipedia.org/wiki/Alan_Turing}{Alan Turing} 
fueron clave para poder descifrar los códigos nazis y garantizar la victoria de
los aliados, por saber los planes de los alemanes y la ubicación de sus barcos, submarinos y suministros. De los trabajos de Turing y Von Neumann en aquella época quedaron los dos modelos 
que sigue toda la computación moderna, ya sean PC's, servidores o smartphones: La
\href{https://es.wikipedia.org/wiki/M\%C3\%A1quina_de_Turing}{máquina de Turing}
y la
\href{https://es.wikipedia.org/wiki/Arquitectura_de_von_Neumann}{arquitectura de Von Neumann}.
Dé esta segunda parte podemos afirmar que \emph{la computación y las matemáticas ayudaron a los
los aliados a derrotar a los nazis}.
\newline\newline
Durante esa misma época, alrededor del año 1946 el mismo Von Neumann estuvo en
\href{https://es.wikipedia.org/wiki/Laboratorio_Nacional_de_Los_\%C3\%81lamos}{Los Alamos} 
(el centro de investigación que desarrolló la bomba atómica), formando parte del 
equipo que desarrolló el Método de Montecarlo, considerado como el algoritmo
\href{http://www.uta.edu/faculty/rcli/TopTen/topten.pdf}{más importante del siglo XX}.
Éste puede ser utilizado en una gran cantidad de temas, que abarcan desde el análisis financiero y de riesgos, los gráficos computacionales, y la simulación científica, entre muchos otros. No exageramos demasiado al decir que \emph{es una parte importante del desarrollo económico del siglo XX}.
\newline\newline
A ésta sección aún le falta redacción, pero espero que incluso en su estado actual
permita comprender cuál es la misión del libro. La forma en que se quiere llevar a
cabo sigue a continuación.

\subsection{Propósito y estructura del libro}\label{estructura}
\newline
Muchas veces sucede que la gente escribe los libros que les hubiese gustado leer,
y ésta no es una excepción. El lector ideal es, por lo tanto, un estudiante que
ha pasado por el sistema educativo olvidando casi todo luego de los exámenes por
no haberlo puesto en práctica lo aprendido, y que desea tener en sus manos ese mismo poder que permite
comprender la realidad, descubrir cosas nuevas, construir empresas
y dirigir ejércitos. Por supuesto los detalles son muchos y bastante complejos,
pero la esperanza principal del libro, y la que espero que sea posible encender
y avivar con él, es que todas éstas cosas suelen estar lejos no porque uno no sea
lo suficientemente afortunado o inteligente, sino porque el sistema educativo está
atrozmente obsoleto, y no es capaz de ver más allá de los pénsums oficiales para
distinguir las esperanzas personales de cada estudiante.
\newline\newline
El libro está hecho de la siguiente forma: Empezando desde el álgebra, se describirá
cada concepto necesario de la forma más intuitiva posible, explicando cual es su utilidad
en problemas reales y su posición dentro de las matemáticas. Luego se resolverán 
diferentes ejercicios de ejemplo utilizando librerías externas de Python, y al 
final se explicará un algoritmo 
que permita resolver los mismos problemas. Éste algoritmo probablemente no será
tan eficiente como los presentes en las librerías especializadas, pero se hará
un esfuerzo porque sea lo más legible posible y represente de forma fiel los
conceptos matemáticos que utiliza. De éste modo se espera
cubrir incrementalmente las necesidades de tres tipos de lectores distintos: 
Aquellos que están empezando desde cero, las personas que ya conocen el tema y 
que sólo deseen aprender cómo resolver problemas (estilo cookbook), y aquellos
que deseen comprender el mecanismo que permite resolverlo. El libro está hecho
para que los primeros lean cada capítulo en orden, aprendiendo todo lo necesario
paso a paso, y para que los otros puedan saltar directamente hacia el tema que
deseen. En un futuro se 
espera agregar una cuarta parte con el razonamiento matemático, pero ésto se
hará cuando ya las tres partes anteriores estén realizadas.
\newline\newline
Para terminar, éste libro se encuentra bajo la licencia \href{https://creativecommons.org/licenses/by-nc/3.0/}{Creative Commons BY-NC 3.0}. Puedes redistribuirlo con libertad, crear obras derivadas y citarlo cómo prefieras, siempre y cuando tengas la bondad de mencionarme y que lo hagas sin fines de lucro.

\subsection{Contenido}

El libro inicia describiendo el lenguaje de Programación Python, y proporcionando instrucciones sobre como instalarlo junto con las librerías necesarias en sistemas Windows y Linux. Entre los tópicos a ser tratados se encuentran álgebra (elemental y lineal), cálculo (límites, derivadas, integrales y diferenciales), estadísticas, grafos (redes sociales), métodos numéricos y aprendizaje automático (machine learning).

\subsection{Pre-requisitos}

El único conocimiento necesario para seguir el libro son las cuatro operaciones
aritméticas elementales (suma, resta, multiplicación y división). Tampoco haría
daño tener vagos recuerdos sobre cómo despejar ecuaciones elementales. Es 
conveniente notar aquí que el autor está releyendo sobre varias conceptos de
secundaria para escribir sobre ellos, y al momento de iniciar el libro (por ahora)
tampoco conoce las técnicas más avanzadas.

\subsection{Referencias (temporal)}

Qué es programar: \href{http://gr3p.com/2012/10/que-es-programar-iii/}{gr3p blog}\newline
Aprender Python: \href{http://learnpythonthehardway.org/}{Learn Python the Hard Way}\newline
Matemática mal enseñada: \href{http://www.maa.org/devlin/lockhartslament.pdf}{Lockhart's Lament}\newline
Cálculo: \href{http://www.gutenberg.org/ebooks/33283}{Calculus Made Easy}\newline
Ecuaciones diferenciales: \href{http://mysite.science.uottawa.ca/rsmith43/Zombies.pdf}{Zombie Breakout}\newline

\label{end}\end{document}
